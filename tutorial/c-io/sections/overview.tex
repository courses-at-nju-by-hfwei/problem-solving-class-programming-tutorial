\section{课程简介}

%%%%%%%%%%%%%%%
\begin{frame}{警告:不要在周五的晚上写代码!}
  \fignocaption{width = 0.35\textwidth}{figs/code-joke.jpg}
\end{frame}
%%%%%%%%%%%%%%%

%%%%%%%%%%%%%%%
% \begin{frame}{}
%   \begin{columns}
%     \column{0.50\textwidth}
%       \fignocaption{width = 0.85\textwidth}{figs/embrace-bug.jpg}
%     \column{0.50\textwidth}
%       \fignocaption{width = 0.65\textwidth}{figs/keep-calm-code.png}
%   \end{columns}
% \end{frame}
%%%%%%%%%%%%%%%

%%%%%%%%%%%%%%%
\begin{frame}{辅导课程介绍:C or C++}
  \begin{columns}
    \column{0.45\textwidth}
      C 语言是基础:
      \begin{itemize}
	\item IO
	\item Control Flow
	\item Function
	\item Array (String) \& Pointers
	\item Struct
      \end{itemize}
    \column{0.55\textwidth}
      C++ 语言的特性:
      \begin{itemize}
	\item OO (Object-oriented)
	\item Templates
	\item STL (Standard Template Library)
	\item FP (Functional Programming)
      \end{itemize}
  \end{columns}

  \vspace{1.00cm}
  \centerline{先学习 C 语言; 是否学习 C++ 视情况而定。}
\end{frame}
%%%%%%%%%%%%%%%

%%%%%%%%%%%%%%%
\begin{frame}{辅导课程介绍:课程形式}
  \begin{description}
    \setlength{\itemsep}{8pt}
    \item[讲解] 语言知识点、常见的``坑'' (点到为止)
    \item[练习] 指定 OJ题目: 互助练习、从旁辅导
    \item[分享] 得意之处、Debug 之痛
    \pause
    \item[课后] 继续完成 OJ 剩余题目
  \end{description}
\end{frame}
%%%%%%%%%%%%%%%

%%%%%%%%%%%%%%%
\begin{frame}{C 语言的历史、标准与实现}
  \centerline{1972年, Dennis Ritchie 创造了 C 语言:}
  \fignocaption{width = 0.20\textwidth}{figs/Ritchie.jpg}

  \vspace{0.50cm}
  \centerline{K\&R C (1978) $\to$ ANSI C/ISO C (C89/C90) $\to$ C99 $\to$ C11}

  \vspace{0.30cm}
  \centerline{GCC, Clang, Microsoft Visual C++, $\ldots$}
\end{frame}
%%%%%%%%%%%%%%%

%%%%%%%%%%%%%%%
\begin{frame}{参考资料}
  \begin{columns}
    \column{0.50\textwidth}
      \fignocaption{width = 0.60\textwidth}{figs/kandr-2nd.png}
    \column{0.50\textwidth}
      \fignocaption{width = 0.55\textwidth}{figs/kandr-hello-world.jpg}{\centerline{Hello World!}}
  \end{columns}

  \vspace{0.20cm}
  \pause
  \begin{center}
    \url{http://www.cplusplus.com/}\\[8pt]
    \url{http://en.cppreference.com/w/}
  \end{center}
\end{frame}
%%%%%%%%%%%%%%%

%%%%%%%%%%%%%%%
\begin{frame}{听说你还没有注册过 $\cdots$}
  \fignocaption{width = 0.60\textwidth}{figs/stackoverflow-logo.png}
\end{frame}
%%%%%%%%%%%%%%%