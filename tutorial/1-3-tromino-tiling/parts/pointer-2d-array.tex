%%%%%%%%%%%%%%%
\begin{frame}{}
  \centerline{\LARGE 2D Arrays}
\end{frame}
%%%%%%%%%%%%%%%

%%%%%%%%%%%%%%%
\begin{frame}[fragile]{}
  \begin{lstlisting}[style = Cstyle]
    int a[3][5] = {
      {1,2,3,4,5}, 
      {6,7,8,9,10},
      {11,12,13}
    }; 
  \end{lstlisting}

  \vspace{0.40cm}
  \pause
  \begin{quote}
    Elements (of an 2D array) are stored by rows.

    \hfill --- K\&R
  \end{quote}

  \vspace{0.80cm}
  \centerline{\teal{\texttt{array-2d.c (Part I)}}}
\end{frame}
%%%%%%%%%%%%%%%

%%%%%%%%%%%%%%%
\begin{frame}[fragile]{}
  \begin{quote}
    In C, a 2D array is really a 1D array,
    each of whose elements is an array.

    \hfill --- K\&R
  \end{quote}

  \vspace{0.50cm}
  \pause
  \begin{lstlisting}[style = Cstyle]
  a, $\quad$ &a[0], $\quad$ a[0], $\quad$ &a[0][0], $\quad$ &a
  int (*pa)[5] = a; // a pointer to an array of 5 integers
  \end{lstlisting}

  \vspace{0.30cm}
  \centerline{\teal{\texttt{array-2d.c (Part II)}}}

  \vspace{0.50cm}
  \pause
  \begin{lstlisting}[style = Cstyle]
  a[i][j]  // *((*(a + i)) + j)
  \end{lstlisting}
\end{frame}
%%%%%%%%%%%%%%%

%%%%%%%%%%%%%%%
\begin{frame}[fragile]{}
  \begin{lstlisting}[style = Cstyle]
  void f(int a[3][5]);
  void f(int a[][5], int m); // m rows
  void f(int (*a)[5], int m);

  f(a, 3);  // int a[3][5];
  f(pa, 3); // int (*pa)[5] = a;
  \end{lstlisting}
\end{frame}
%%%%%%%%%%%%%%%

%%%%%%%%%%%%%%%
\begin{frame}[fragile]{}
  \begin{lstlisting}[style = Cstyle]
  void f(int m, int n, int a[m][n]);
  void f(int m, int n, int a[][n]);
  void f(int m, int n, int (*a)[n]);

  int a[m][n];
  f(m, n, a);

  int (*pa)[n] = malloc( sizeof(int[m][n]) );
  f(m, n, pa);
  \end{lstlisting}
\end{frame}
%%%%%%%%%%%%%%%
