\section{IO}

%%%%%%%%%%%%%%%
\begin{frame}{IO 基本概念}
  IO 模型:字符流 (character stream)
  \begin{itemize}
    \item a sequence of characters divided into lines
    \item each line consists of $\ge 0$ characters followed by a newline character 
  \end{itemize}

  \vspace{0.80cm}
  \centerline{\texttt{\#include <stdio.h>}}
\end{frame}
%%%%%%%%%%%%%%%

%%%%%%%%%%%%%%%
\begin{frame}{字符输入输出: \texttt{getchar}, \texttt{putchar}}
  \begin{center}
    \href{http://www.cplusplus.com/reference/cstdio/getchar/}{\texttt{int \blue{getchar}(void);}} \\[8pt]
    \href{http://www.cplusplus.com/reference/cstdio/putchar/}{\texttt{int \blue{putchar}(int character);}}
  \end{center}

  \vspace{0.50cm}
  \centerline{代码示例: \texttt{file-copy.c}}

  \pause
  \vspace{0.50cm}
  \centerline{代码运行:\texttt{gcc file-copy.c -o file-copy \;\;  ./file-copy}}

  \pause
  \vspace{0.80cm}
  \centerline{代码运行:\texttt{./file-copy <file-copy-file} (OJ 测试用例)}
\end{frame}
%%%%%%%%%%%%%%%

%%%%%%%%%%%%%%%
\begin{frame}{EOF}
  \href{http://www.cplusplus.com/reference/cstdio/EOF/}{EOF} (文件结束符):由宏定义的整型数
  \begin{itemize}
    \item EOF 是整型数 (\texttt{int c}; \texttt{!= EOF})
    \item EOF 不是实际字符, 不同于行结束符 ``\texttt{\textbackslash n}'' (\texttt{== '\textbackslash n'})
    \item EOF 不单单指示``文件''的结束, 而是指示字符流的结束
  \end{itemize}

  \begin{description}[Windows]
    \item[Windows] \texttt{Ctrl + Z}
    \item[Linux] \texttt{Ctrl + D}
    \item[Mac] \texttt{Ctrl + D}
  \end{description}
\end{frame}
%%%%%%%%%%%%%%%

%%%%%%%%%%%%%%%
\begin{frame}{格式化输出:\texttt{printf}}
  \centerline{\href{http://www.cplusplus.com/reference/cstdio/printf/}{\texttt{int \blue{printf}(const char* format, ...);}}}

  \vspace{0.50cm}
  \centerline{format: \texttt{\%[flags][width][.precision][length]specifier}}

  \vspace{0.60cm}
  \pause
  \begin{columns}
    \column{0.50\textwidth}
      常用的输出格式:
      \begin{description}[Decimal integer]
	\item[Decimal integer] \texttt{\%d, \%ld}
	\item[Decimal float] \texttt{\%f, \%.2f}
	\item[Character] \texttt{\%c}
	\item[String] \texttt{\%s, \%.5s}
      \end{description}
    \column{0.50\textwidth}
      \centerline{代码示例:\texttt{printf.c}}
  \end{columns}

  \vspace{0.30cm}
  \centerline{使用 \texttt{printf(''\%s'', s)}, 不要使用 \texttt{printf(s)}。}
\end{frame}
%%%%%%%%%%%%%%%

%%%%%%%%%%%%%%%
\begin{frame}{其它格式化输出函数:\texttt{fprintf}, \texttt{sprintf}}
  \begin{center}
    \href{http://www.cplusplus.com/reference/cstdio/fprintf/}
    {\texttt{int \blue{fprintf}(FILE *stream, const char *format, ...);}} \\[0.50cm]

    \href{http://www.cplusplus.com/reference/cstdio/sprintf/}
    {\texttt{int \blue{sprintf}(char *str, const char *format, ...);}}
  \end{center}
\end{frame}
%%%%%%%%%%%%%%%

%%%%%%%%%%%%%%%
\begin{frame}{格式化输入:\texttt{scanf}}
  \centerline{\href{http://www.cplusplus.com/reference/cstdio/scanf/}
  {\texttt{int \blue{scanf}(const char *format, ...)}}}

  \vspace{0.30cm}
  \begin{enumerate}
    \item Read characters from the standard input
    \item Interpret according to \texttt{format}
    \item Store in arguments (pointers)
  \end{enumerate}

  \vspace{0.30cm}
  \pause
  \texttt{format}:
  \begin{itemize}
    \item Whitespace character
    \item Non-whitespace character, except format specifier (\%)
    \item Format specifiers: \texttt{\%[*][width][length]specifier}
  \end{itemize}
\end{frame}
%%%%%%%%%%%%%%%

%%%%%%%%%%%%%%%
\begin{frame}{格式化输入:\texttt{scanf}}
  \centerline{代码示例:\texttt{scanf.c}}

  \vspace{0.60cm}
  Return value:
  \begin{description}
    \item[Success] \# of items of the argument list successfully filled
      \begin{itemize}
	\item $\le$ \# of arguments
      \end{itemize}
    \item[Failure] EOF
      \begin{itemize}
	\item reading error (\texttt{ferror}), end-of-file (\texttt{feof})
      \end{itemize}
  \end{description}
\end{frame}
%%%%%%%%%%%%%%%

%%%%%%%%%%%%%%%
\begin{frame}{其它格式化输入函数: \texttt{fscanf}, \texttt{sscanf}}
  \begin{center}
    \href{http://www.cplusplus.com/reference/cstdio/fscanf/}
    {\texttt{int \blue{fscanf}(FILE *stream, const char *format, ...);}} \\[0.50cm]

    \href{http://www.cplusplus.com/reference/cstdio/sscanf/}
    {\texttt{int \blue{sscanf}(const char *s, const char *format, ...);}}
  \end{center}
\end{frame}
%%%%%%%%%%%%%%%

%%%%%%%%%%%%%%%
\begin{frame}{行输入输出}
  \begin{center}
    \href{http://www.cplusplus.com/reference/cstdio/fgets/}
    {\texttt{char *\blue{fgets}(char *str, int num, FILE *stream);}}  \\[0.30cm]

    \href{http://www.cplusplus.com/reference/cstdio/gets/}{\texttt{\textcolor{gray}{char *gets(char *str);}}}  \\[0.50cm]

    \href{http://www.cplusplus.com/reference/cstdio/puts/}
    {\texttt{int \blue{puts}(const char *str);}}
  \end{center}
\end{frame}
%%%%%%%%%%%%%%%

%%%%%%%%%%%%%%%
\begin{frame}{OJ 练习之 IO}
  \centerline{OJ 练习示例 ($1.1.1)$}

  \vspace{0.80cm}
  \pause
  \[
    1.1.1 \sim 1.1.6 \qquad 1.2.2
  \]
\end{frame}
%%%%%%%%%%%%%%%

%%%%%%%%%%%%%%%
\begin{frame}{OJ 常见输入模式}
  \centerline{IO 练习之总结与分享}
\end{frame}
%%%%%%%%%%%%%%%

%%%%%%%%%%%%%%%
% \begin{frame}{Problem $1.1.1$}
%   \begin{description} 
%     \item[Problem]
%      Your task is to Calculate a + b.
%    \item[Input]
%      The input will consist of a series of pairs of integers a and b,
%      separated by a space, one pair of integers per line.
%    \item[Sample Input]
%      \begin{align*}
%        1 \;& 5 \\
%        10 & 20
%      \end{align*}
%    \item[Output]
%      For each pair of input integers a and b you should output the sum of a and b in one line,
%      and with one line of output for each line in input.
%    \item[Sample Output]
%      \begin{align*}
%        &6 \\
%        &30
%      \end{align*}
%    \end{description}
% \end{frame}
%%%%%%%%%%%%%%%
