\section{IO}

%%%%%%%%%%%%%%%
\begin{frame}{IO 基本概念}
  IO 模型:字符流 (character stream)
  \begin{itemize}
    \item a sequence of characters divided into lines
    \item each line consists of $\ge 0$ characters followed by a newline character 
  \end{itemize}
\end{frame}
%%%%%%%%%%%%%%%

%%%%%%%%%%%%%%%
\begin{frame}{Standard Input and Output: 以 \texttt{getchar}, \texttt{putchar} 为例}
  \centerline{代码示例: \texttt{file-copy.c}}

  \vspace{0.50cm}
  \pause
  \href{http://www.cplusplus.com/reference/cstdio/EOF/}{EOF} (文件结束符):
  \begin{itemize}
    \item EOF 不单单指示``文件''的结束,而是指示字符流的结束
    \item EOF 不同于 (行结束)
    \item EOF 不是实际字符, 而是一个宏定义 (\texttt{!= EOF})
    \item EOF 是整型数 (\texttt{int c}) 
  \end{itemize}

  \pause
  \vspace{0.50cm}
  \centerline{代码运行:\texttt{./file-copy}}

  \pause
  \begin{columns}[c]
    \column{0.45\textwidth}
      \begin{description}[Windows]
	\item[Windows] \texttt{Ctrl + Z}
	\item[Linux] \texttt{Ctrl + D}
	\item[Mac] \texttt{Ctrl + D}
      \end{description}
    \column{0.45\textwidth}
      请仔细观察并思考:
      \begin{enumerate}
	\item \texttt{Enter} 的作用
	\item EOF 的作用
      \end{enumerate}
  \end{columns}
\end{frame}
%%%%%%%%%%%%%%%

%%%%%%%%%%%%%%%
\begin{frame}{IO on OJ: input redirection (输入重定向)}
  \centerline{代码运行:\texttt{./file-copy <file-copy-file}}
\end{frame}
%%%%%%%%%%%%%%%

%%%%%%%%%%%%%%%
\begin{frame}{格式化输出:\texttt{printf}}
  \centerline{\href{http://www.cplusplus.com/reference/cstdio/printf/}{\texttt{int printf(const char* format, ...);}}}

  \vspace{0.50cm}
  \centerline{format: \texttt{\%[flags][width][.precision][length]specifier}}

  \vspace{0.60cm}
  \pause
  \begin{columns}
    \column{0.50\textwidth}
      常用的输出格式:
      \begin{description}[Decimal integer]
	\item[Decimal integer] \texttt{\%d, \%ld}
	\item[Decimal float] \texttt{\%f, \%.2f}
	\item[Character] \texttt{\%c}
	\item[String] \texttt{\%s, \%.5s}
      \end{description}
    \column{0.50\textwidth}
      \centerline{代码示例:\texttt{printf.c}}
  \end{columns}

  \vspace{0.30cm}
  \centerline{使用 \texttt{printf(''\%s'', s)}, 不要使用 \texttt{printf(s)}。}
\end{frame}
%%%%%%%%%%%%%%%

%%%%%%%%%%%%%%%
\begin{frame}{其它输出函数:\texttt{fprintf}, \texttt{sprintf}}
  \centerline{\href{http://www.cplusplus.com/reference/cstdio/fprintf/}
  {\texttt{int fprintf(FILE *stream, const char *format, ...);}}}

  \vspace{0.50cm}
  \centerline{\href{http://www.cplusplus.com/reference/cstdio/sprintf/}
  {\texttt{int sprintf(char *str, const char *format, ...);}}}
\end{frame}
%%%%%%%%%%%%%%%

%%%%%%%%%%%%%%%
\begin{frame}{格式化输入:\texttt{scanf}}
  \centerline{\href{http://www.cplusplus.com/reference/cstdio/scanf/}
  {\texttt{int scanf(const char *format, ...)}}}

  \vspace{0.30cm}
  \begin{enumerate}
    \item Read characters from the standard input
    \item Interpret according to \texttt{format}
    \item Store in arguments (pointers)
  \end{enumerate}

  \vspace{0.30cm}
  \pause
  \texttt{format}:
  \begin{itemize}
    \item Whitespace character
    \item Non-whitespace character, except format specifier (\%)
    \item Format specifiers: \texttt{\%[*][width][length]specifier}
  \end{itemize}
\end{frame}
%%%%%%%%%%%%%%%

%%%%%%%%%%%%%%%
\begin{frame}{格式化输入:\texttt{scanf}}
  \centerline{代码示例:\texttt{scanf.c}}
  % \begin{center}
  %   \texttt{scanf(''\%d \%s \%d'', \&day, \&month\_name, \&year)} \\[6pt]
  %   \texttt{scanf(''\%d, \%s, \%d'', \&day, \&month\_name, \&year)} \\[6pt]
  %   \texttt{scanf(''\%d/\%s/\%d'', \&day, \&month\_name, \&year)}
  % \end{center}

  \vspace{0.60cm}
  Return value:
  \begin{description}
    \item[Success] \# of items of the argument list successfully filled
      \begin{itemize}
	\item $\le$ \# of arguments
      \end{itemize}
    \item[Failure] EOF
      \begin{itemize}
	\item reading error (\texttt{ferror}), end-of-file (\texttt{feof})
      \end{itemize}
  \end{description}
\end{frame}
%%%%%%%%%%%%%%%

%%%%%%%%%%%%%%%
\begin{frame}{其它输入函数: \texttt{fscanf}, \texttt{sscanf}}
  \begin{center}
    \href{}{} 

    \href{}{}
  \end{center}
\end{frame}
%%%%%%%%%%%%%%%

%%%%%%%%%%%%%%%
\begin{frame}{Line Input and Output}
  \begin{center}
    \href{}{\text{}}

    \href{}{\text{gets}} deprecated

    \href{http://www.cplusplus.com/reference/cstdio/puts/}
    {\texttt{int puts(const char *str);}}
  \end{center}
\end{frame}
%%%%%%%%%%%%%%%

%%%%%%%%%%%%%%%
\begin{frame}{OJ 练习之 IO}
  \centerline{OJ 示例:1.1.1}

  \vspace{0.80cm}
  \pause
  \centerline{1.1.2 $\sim$ 1.1.6}
\end{frame}
%%%%%%%%%%%%%%%

%%%%%%%%%%%%%%%
\begin{frame}{OJ 常见输入模式}
  \centerline{IO 练习之总结与分享}
\end{frame}
%%%%%%%%%%%%%%%
